\documentclass[10pt,landscape]{article}
\usepackage{multicol}
\usepackage{calc}
\usepackage{ifthen}
\usepackage[landscape]{geometry}
\usepackage[utf8]{inputenc}
\usepackage[T1]{fontenc}
\usepackage[ngerman]{babel}
\usepackage{amssymb}

% To make this come out properly in landscape mode, do one of the following
% 1.
%  pdflatex latexsheet.tex
%
% 2.
%  latex latexsheet.tex
%  dvips -P pdf  -t landscape latexsheet.dvi
%  ps2pdf latexsheet.ps


% If you're reading this, be prepared for confusion.  Making this was
% a learning experience for me, and it shows.  Much of the placement
% was hacked in; if you make it better, let me know...


% 2008-04
% Changed page margin code to use the geometry package. Also added code for
% conditional page margins, depending on paper size. Thanks to Uwe Ziegenhagen
% for the suggestions.

% 2006-08
% Made changes based on suggestions from Gene Cooperman. <gene at ccs.neu.edu>


% To Do:
% \listoffigures \listoftables
% \setcounter{secnumdepth}{0}


% This sets page margins to .5 inch if using letter paper, and to 1cm
% if using A4 paper. (This probably isn't strictly necessary.)
% If using another size paper, use default 1cm margins.
\ifthenelse{\lengthtest { \paperwidth = 11in}}
	{ \geometry{top=.5in,left=.5in,right=.5in,bottom=.5in} }
	{\ifthenelse{ \lengthtest{ \paperwidth = 297mm}}
		{\geometry{top=1cm,left=1cm,right=1cm,bottom=1cm} }
		{\geometry{top=1cm,left=1cm,right=1cm,bottom=1cm} }
	}

% Turn off header and footer
\pagestyle{empty}
 

% Redefine section commands to use less space
\makeatletter
\renewcommand{\section}{\@startsection{section}{1}{0mm}%
                                {-1ex plus -.5ex minus -.2ex}%
                                {0.5ex plus .2ex}%x
                                {\normalfont\large\bfseries}}
\renewcommand{\subsection}{\@startsection{subsection}{2}{0mm}%
                                {-1explus -.5ex minus -.2ex}%
                                {0.5ex plus .2ex}%
                                {\normalfont\normalsize\bfseries}}
\renewcommand{\subsubsection}{\@startsection{subsubsection}{3}{0mm}%
                                {-1ex plus -.5ex minus -.2ex}%
                                {1ex plus .2ex}%
                                {\normalfont\small\bfseries}}
\makeatother

% Define BibTeX command
\def\BibTeX{{\rm B\kern-.05em{\sc i\kern-.025em b}\kern-.08em
    T\kern-.1667em\lower.7ex\hbox{E}\kern-.125emX}}

% Don't print section numbers
\setcounter{secnumdepth}{0}


\setlength{\parindent}{0pt}
\setlength{\parskip}{0pt plus 0.5ex}


% -----------------------------------------------------------------------

\begin{document}

\raggedright
\footnotesize
\begin{multicols}{3}


% multicol parameters
% These lengths are set only within the two main columns
%\setlength{\columnseprule}{0.25pt}
\setlength{\premulticols}{1pt}
\setlength{\postmulticols}{1pt}
\setlength{\multicolsep}{1pt}
\setlength{\columnsep}{2pt}

\begin{center}
     \Large{\textbf{\LaTeXe\ Befehlsübersicht}} \\
\end{center}

\section{KOMA-Dokumentklassen}
Optimiert für deutsch Dokumente\\
\begin{tabular}{@{}ll@{}}
\verb!scrbook!    & Standard Doppelseitig. \\
\verb!scrreprt!  & Kein \verb!\part! \\
\verb!scrartcl! & Kein \verb!\part! oder \verb!\chapter!
\end{tabular}

Ein Dokument beginnt immer mit
\verb!\documentclass{!\textit{klasse}\verb!}!  Inhalt zwischen
\verb!\begin{document}! und \verb!\end{document}!


\subsection{Wichtige \texttt{documentclass} Optionen}
\newlength{\MyLen}
\settowidth{\MyLen}{\texttt{letterpaper}/\texttt{a4paper} \ }
\begin{tabular}{@{}p{\the\MyLen}%
                @{}p{\linewidth-\the\MyLen}@{}}
\texttt{10pt}/\texttt{11pt}/\texttt{12pt} & Schriftgröße. \\
\texttt{letterpaper}/\texttt{a4paper} & Papierformat. \\
\texttt{twocolumn} & Zweispaltiges Dokument. \\
\texttt{twoside}   & Seitenränder für Doppelseiten. \\
\texttt{landscape} & Querformat.  Dazu muss
                     \texttt{dvips -t landscape} genutzt werden. \\
%\texttt{draft}     & Double-space lines.
\texttt{parskip}     & Freizeile statt Einzug.
\end{tabular}

Benutzung:
\verb!\documentclass[!\textit{opt,opt}\verb!]{!\textit{Klasse}\verb!}!.


\subsection{Wichtige Pakete}
\settowidth{\MyLen}{\texttt{multicol} }
\begin{tabular}{@{}p{1.7cm}%
                @{}p{\linewidth-1.7cm}@{}}
%\begin{tabular}{@{}ll@{}}
\texttt{inputenc}  &  Für direkte Eingabe von Umlauten mit \texttt{[utf8]}. \\
\texttt{fontenc}   &  Darstellung von Umlauten mit \texttt{[T1]}.            \\
\texttt{babel}   &  Neue dt. Rechtschreibung mit \texttt{[ngerman]}.            \\
\texttt{multicol}  &  Verwende $n$ Spalten: \verb!\begin{multicols}{!$n$\verb!}!.   \\
\texttt{amsmath}  &  Erweiterung für \LaTeX{}-Mathe. \\
\texttt{amssymb}     &    Zusätzliche Mathesymbole (bsp. $\mathbb{R}$)\\
\texttt{graphicx}  &  Bilder einbinden:\verb!\includegraphics[width=!%
                        \textit{x}\verb!]{!%
                        \textit{bild}\verb!}!. \\
\texttt{url}       & URLs einfügen: \verb!\url{!%
                        \textit{http://\ldots}%
                        \verb!}!.\\
\texttt{textcomp}       & Zusätzliche Symbole z.B. \verb!\textmu!.\\ %
\texttt{upgreek}    &  Aufrechte griechische Symbole\\
\texttt{wrapfig}    &    Textumflossene Abbildungen\\
\texttt{subcaption}    &    Teilabbildungen in \verb!\begin{figure}! mit \verb!\begin{subfigure}[b]{!\textit{weite}\verb!}! und \verb!\hfill!\\
\texttt{pdfpages}    &    Einbinden von externen PDFs mit \verb!\includepdf[!\textit{pages=1-2}\verb!]{!\textit{Anhang.pdf}\verb!}!\\
\texttt{hyperref}    &    PDF Verlinkungen im Inhaltsverzeichnis
\end{tabular}

Vor \verb!\begin{document}! einbinden mit 
\verb!\usepackage{!\textit{paket}\verb!}!


\subsection{Titel}
\settowidth{\MyLen}{\texttt{.author.text.} }
\begin{tabular}{@{}p{\the\MyLen}%
                @{}p{\linewidth-\the\MyLen}@{}}
\verb!\author{!\textit{text}\verb!}! & Autor des Dokuments. \\
\verb!\title{!\textit{text}\verb!}!  & Titel des Dokuments. \\
\verb!\date{!\textit{text}\verb!}!   & Datum. \\
\end{tabular}

Diese Befehle stehen vor \verb!\begin{document}!.  Mit
\verb!\maketitle! am Anfang des Dokuments kann die Titelseite erzeugt werden.

\subsection{Verzeichnisse}
\settowidth{\MyLen}{\texttt{tableofcontentsund} \ }
\begin{tabular}{@{}p{\the\MyLen}@{}p{\linewidth-\the\MyLen}@{}}
\verb!\tableofcontents!    &    Inhaltsverzeichnis \\
\verb!\listoffigures!    &    Abbildungsverzeichnis \\
\verb!\listoftables!    &    Tabellenverzeichnis
\end{tabular}


%\subsection{Sonstiges}
%\settowidth{\MyLen}{\texttt{.pagestyle.empty.} }
%\begin{tabular}{@{}p{\the\MyLen}%
%                @{}p{\linewidth-\the\MyLen}@{}}
%\verb!\pagestyle{empty}!     &   Keine Kopf-,Fusszeile und Seitennummer. \\
%
%\end{tabular}



\section{Dokumentstruktur}
\begin{multicols}{2}
\verb!\part{!\textit{titel}\verb!}!  \\
\verb!\chapter{!\textit{titel}\verb!}!  \\
\verb!\section{!\textit{titel}\verb!}!  \\
\verb!\subsection{!\textit{titel}\verb!}!  \\
\verb!\subsubsection{!\textit{titel}\verb!}!  \\
\verb!\paragraph{!\textit{titel}\verb!}!  \\
\verb!\subparagraph{!\textit{titel}\verb!}!
\end{multicols}
{\raggedright
Überschriften mit einem \texttt{*} am Ende, wie
\verb!\section*{!\textit{titel}\verb!}!, tragen keine Nummerierung. \verb!\setcounter{secnumdepth}{!$x$\verb!}! unterdrückt Nummerierung ab Tiefe $>x$, gezählt von \verb!chapter! mit Tiefe 0.
}

\subsection{Textumgebungen}
\settowidth{\MyLen}{\texttt{.begin.quotation.}}
\begin{tabular}{@{}p{\the\MyLen}%
                @{}p{\linewidth-\the\MyLen}@{}}
\verb!\begin{comment}!    &  Kommentar (wird nicht im Text angezeigt). \\
\verb!\begin{quote}!      &  Eingerückter Zitatblock. \\
\verb!\begin{verse}!      &  Zitat für Verse.
\end{tabular}

\subsection{Listen}
\settowidth{\MyLen}{\texttt{.begin.description.}}
\begin{tabular}{@{}p{\the\MyLen}%
                @{}p{\linewidth-\the\MyLen}@{}}
\verb!\begin{enumerate}!        &  Nummerierte Liste. \\
\verb!\begin{itemize}!          &  Gepunktete Liste. \\
\verb!\begin{description}!      &  Beschreibungsliste. \\
\verb!\item! \textit{text}      &  Listenelement. \\
\verb!\item[!\textit{x}\verb!]! \textit{text}
                                & \textit{x} anstatt der Nummer oder des Punktes verwenden. Pflicht für \verb!description!. \\
\end{tabular}




\subsection{Querverweis}
\settowidth{\MyLen}{\texttt{.pageref.marker..}}
\begin{tabular}{@{}p{\the\MyLen}%
                @{}p{\linewidth-\the\MyLen}@{}}
\verb!\label{!\textit{name}\verb!}!   &  Setze Markierung für Querverweis, 
                          oft in der Form \verb!\label{sec:item}!. \\
\verb!\ref{!\textit{name}\verb!}!   &  Gibt die Nummer am Markierer. \\
\verb!\eqref{!\textit{name}\verb!}! &  Gibt die Formelnummer in Klammern (1.1) \\
\verb!\pageref{!\textit{name}\verb!}! &  Gibt die Seite am Markierer. \\
\verb!\footnote{!\textit{text}\verb!}!  &  Gibt eine Fußnote aus. \\
\end{tabular}


\subsection{Fließumgebungen}
\settowidth{\MyLen}{\texttt{.begin.equation..place.}}
\begin{tabular}{@{}p{\the\MyLen}%
                @{}p{\linewidth-\the\MyLen}@{}}
\verb!\begin{table}[!\textit{ort}\verb!]!     &  Nummerierte Tabelle. \\
\verb!\begin{figure}[!\textit{ort}\verb!]!    &  Nummerierte Abbildung. \\
\verb!\begin{equation}[!\textit{ort}\verb!]!  &  Nummerierte Formel. \\
\verb!\caption{!\textit{text}\verb!}!           &  Unterschrift hinzufügen. \\
\end{tabular}

The \textit{place} is a list valid placements for the body.  \texttt{t}=top,
\texttt{h}=here, \texttt{b}=bottom, \texttt{p}=separate page, \texttt{!}=place even if ugly.  Captions and label markers should be within the environment.

%---------------------------------------------------------------------------

\section{Text Eigenschaften}

\subsection{Schriftart}
\newcommand{\FontCmd}[3]{\PBS\verb!\#1{!\textit{text}\verb!}!  \> %
                         \verb!{\#2 !\textit{text}\verb!}! \> %
                         \#1{#3}}
\begin{tabular}{@{}l@{}l@{}l@{}}
\textit{Befehl} & \textit{Deklaration} & \textit{Effekt} \\
\verb!\textrm{!\textit{text}\verb!}!                    & %
        \verb!{\rmfamily !\textit{text}\verb!}!               & %
        \textrm{Roman family} \\
\verb!\textsf{!\textit{text}\verb!}!                    & %
        \verb!{\sffamily !\textit{text}\verb!}!               & %
        \textsf{Sans serif family} \\
\verb!\texttt{!\textit{text}\verb!}!                    & %
        \verb!{\ttfamily !\textit{text}\verb!}!               & %
        \texttt{Typewriter family} \\
\verb!\textmd{!\textit{text}\verb!}!                    & %
        \verb!{\mdseries !\textit{text}\verb!}!               & %
        \textmd{Medium series} \\
\verb!\textbf{!\textit{text}\verb!}!                    & %
        \verb!{\bfseries !\textit{text}\verb!}!               & %
        \textbf{Bold series} \\
\verb!\textup{!\textit{text}\verb!}!                    & %
        \verb!{\upshape !\textit{text}\verb!}!               & %
        \textup{Upright shape} \\
\verb!\textit{!\textit{text}\verb!}!                    & %
        \verb!{\itshape !\textit{text}\verb!}!               & %
        \textit{Italic shape} \\
\verb!\textsl{!\textit{text}\verb!}!                    & %
        \verb!{\slshape !\textit{text}\verb!}!               & %
        \textsl{Slanted shape} \\
\verb!\textsc{!\textit{text}\verb!}!                    & %
        \verb!{\scshape !\textit{text}\verb!}!               & %
        \textsc{Small Caps shape} \\
\verb!\emph{!\textit{text}\verb!}!                      & %
        \verb!{\em !\textit{text}\verb!}!               & %
        \emph{Emphasized} \\
\verb!\textnormal{!\textit{text}\verb!}!                & %
        \verb!{\normalfont !\textit{text}\verb!}!       & %
        \textnormal{Standardschrift} \\
\verb!\underline{!\textit{text}\verb!}!                 & %
                                                        & %
        \underline{Underline}
\end{tabular}

Die Befehle (Beispiel: t\textit{tt}t) produzieren die besseren Abstände als die Deklarationen (Beispiel: t{\itshape tt}t) form.

\subsection{Schriftgröße}
\setlength{\columnsep}{14pt} % Need to move columns apart a little
\begin{multicols}{2}
\begin{tabbing}
\verb!\footnotesize!          \= \kill
\verb!\tiny!                  \>  \tiny{tiny} \\
\verb!\scriptsize!            \>  \scriptsize{scriptsize} \\
\verb!\footnotesize!          \>  \footnotesize{footnotesize} \\
\verb!\small!                 \>  \small{small} \\
\verb!\normalsize!            \>  \normalsize{normalsize} \\
\verb!\large!                 \>  \large{large} \\
\verb!\Large!                 \=  \Large{Large} \\  % Tab hack for new column
\verb!\LARGE!                 \>  \LARGE{LARGE} \\
\verb!\huge!                  \>  \huge{huge} \\
\verb!\Huge!                  \>  \Huge{Huge}
\end{tabbing}
\end{multicols}
\setlength{\columnsep}{1pt} % Set column separation back

Diese Deklarationen sollten in der Form
\verb!{\small! \ldots\verb!}! genutzt werden. Ohne Klammern wirken sie im ganzen Dokument.


\subsection{Befehlausblendung}

\settowidth{\MyLen}{\texttt{.begin.verbatim..} }
\begin{tabular}{@{}p{\the\MyLen}%
                @{}p{\linewidth-\the\MyLen}@{}}
\verb@\begin{verbatim}@ & Alles wird als reiner Text gesehen. \\
\verb@\begin{verbatim*}@ & Zeigt Leerzeichen als \verb*@ @. \\
\verb@\verb!text!@ & Wie \verb!verbatim! zwischen den Begrenzungszeichen (in diesem Fall %
                      `\texttt{!}').
\end{tabular}


\subsection{Ausrichtung}
\begin{tabular}{@{}ll@{}}
\textit{Umgebung}  &  \textit{Deklaration}  \\
\verb!\begin{center}!      & \verb!\centering!  \\
\verb!\begin{flushleft}!  & \verb!\raggedright! \\
\verb!\begin{flushright}! & \verb!\raggedleft!  \\
\end{tabular}

\subsection{Sonstiges}
\verb!\linespread{!$x$\verb!}! multipliziert den Zeilenabstand mit $x$.





\section{Textmodus Symbole}

\subsection{Symbole}
\begin{tabular}{@{}l@{\hspace{1em}}l@{\hspace{2em}}l@{\hspace{1em}}l@{\hspace{2em}}l@{\hspace{1em}}l@{\hspace{2em}}l@{\hspace{1em}}l@{}}
\&              &  \verb!\&! &
\_              &  \verb!\_! &
\ldots          &  \verb!\ldots! &
\textbullet     &  \verb!\textbullet! \\
\$              &  \verb!\$! &
\^{}            &  \verb!\^{}! &
\textbar        &  \verb!\textbar! &
\textbackslash  &  \verb!\textbackslash! \\
\%              &  \verb!\%! &
\~{}            &  \verb!\~{}! &
\#              &  \verb!\#! &
\S              &  \verb!\S! \\
\end{tabular}

\subsection{Akzente}
\begin{tabular}{@{}l@{\ }l|l@{\ }l|l@{\ }l|l@{\ }l|l@{\ }l@{}}
\`o   & \verb!\`o! &
\'o   & \verb!\'o! &
\^o   & \verb!\^o! &
\~o   & \verb!\~o! &
\=o   & \verb!\=o! \\
\.o   & \verb!\.o! &
\"o   & \verb!\"o! &
\c o  & \verb!\c o! &
\v o  & \verb!\v o! &
\H o  & \verb!\H o! \\
\c c  & \verb!\c c! &
\d o  & \verb!\d o! &
\b o  & \verb!\b o! &
\t oo & \verb!\t oo! &
\oe   & \verb!\oe! \\
\OE   & \verb!\OE! &
\ae   & \verb!\ae! &
\AE   & \verb!\AE! &
\aa   & \verb!\aa! &
\AA   & \verb!\AA! \\
\o    & \verb!\o! &
\O    & \verb!\O! &
\l    & \verb!\l! &
\L    & \verb!\L! &
\i    & \verb!\i! \\
\j    & \verb!\j! &
!`    & \verb!~`! &
?`    & \verb!?`! &
\end{tabular}


\subsection{Trennzeichen}
\begin{tabular}{@{}l@{\ }ll@{\ }ll@{\ }ll@{\ }ll@{\ }ll@{\ }l@{}}
`       & \verb!`!  &
``      & \verb!``! &
\{      & \verb!\{! &
\lbrack & \verb![! &
(       & \verb!(! &
\textless  &  \verb!\textless! \\
'       & \verb!'!  &
''      & \verb!''! &
\}      & \verb!\}! &
\rbrack & \verb!]! &
)       & \verb!)! &
\textgreater  &  \verb!\textgreater! \\
\end{tabular}

\subsection{Striche}
\begin{tabular}{@{}llll@{}}
\textit{Name} & \textit{Code} & \textit{Beispiel} & \textit{Verwendung} \\
Bindestrich  & \verb!-!   &    Anna-Lisa    & In Worten. \\
Halbgeviertstrich & \verb!--!  & 1--5           & Zwischen Zahlen. \\
Geviertstrich & \verb!---! & Ja---oder nein?    & Interpunktion.
\end{tabular}


\subsection{Zeilen- und Seitenumbrüche}
\settowidth{\MyLen}{\texttt{.pagebreak} }
\begin{tabular}{@{}p{\the\MyLen}%
                @{}p{\linewidth-\the\MyLen}@{}}
\verb!\\!          &  Manueller Zeilenumbruch .  \\
\verb!\\*!         &  Verhindert Seiten- nach Zeilenumbruch. \\
\verb!\kill!       &  Aktuelle Zeile nicht ausgeben. \\
\verb!\pagebreak!  &  Manueller Seitenumbruch. \\
\verb!\noindent!   &  Neue Zeile nicht einrücken.
\end{tabular}


\subsection{Sonstiges}
\settowidth{\MyLen}{\texttt{.rule.w..h.} }
\begin{tabular}{@{}p{\the\MyLen}%
                @{}p{\linewidth-\the\MyLen}@{}}
\verb!\today!  &  \today{} (aktuelles Datum). \\
\verb!$\sim$!  &  Gibt $\sim$ statt \~{} (Code: \verb!\~{}!) aus. \\
\verb!~!       &  Leerzeichen ohne Zeilenumbruch (\verb!W.J.~Clinton!). \\
\verb!\hspace{!$l$\verb!}! & Horizontaler Abstand mit Länge $l$
                                ($l=\mathtt{20pt}$). \\
\verb!\vspace{!$l$\verb!}! & Vertikaler Abstand $l$. \\
\verb!\rule{!$w$\verb!}{!$h$\verb!}! & Linie mit Weiter $w$ und Höhe $h$. \\
\end{tabular}



\section{Tabellenumgebungen}

\subsection{\texttt{tabbing} Umgebung}
\begin{tabular}{@{}l@{\hspace{1.5ex}}l@{\hspace{10ex}}l@{\hspace{1.5ex}}l@{}}
\verb!\=!  &   Setze Tabulator. &
\verb!\>!  &   Gehe zu Tabulator. \\
\verb!\\!    &    Zeilenumbruch & 
\verb!\kill!    &    Zeile nicht ausgeben
\end{tabular}

\subsection{\texttt{tabular} Umgebung}
\verb!\begin{array}[!\textit{pos}\verb!]{!\textit{spalten}\verb!}!   \\
\verb!\begin{tabular}[!\textit{pos}\verb!]{!\textit{spalten}\verb!}! \\
\verb!\begin{tabular*}{!\textit{weite}\verb!}[!\textit{pos}\verb!]{!\textit{spalten}\verb!}!


\subsubsection{\texttt{tabular} Spalten Eigenschaften}
\settowidth{\MyLen}{\texttt{p}\{\textit{width}\} \ }
\begin{tabular}{@{}p{\the\MyLen}@{}p{\linewidth-\the\MyLen}@{}}
\texttt{l}    &   Linksbündige Spalte.  \\
\texttt{c}    &   Zentrierte Spalte.  \\
\texttt{r}    &   Rechtsbündige Spalte. \\
\verb!p{!\textit{width}\verb!}!  &  Genau wie %
                              \verb!\parbox[t]{!\textit{weite}\verb!}!. \\ 
\verb!@{!\textit{decl}\verb!}!   &  Fügt \textit{decl} zwischen die Spalten ein. \\
\verb!|!      &   Fügt eine Linie zwischen die Spalten ein. 
\end{tabular}


\subsubsection{\texttt{tabular} Elemente}
\settowidth{\MyLen}{\texttt{.cline.x-y..}}
\begin{tabular}{@{}p{\the\MyLen}@{}p{\linewidth-\the\MyLen}@{}}
\verb!\hline!           &  Horizontale Linie zwischen Reihen.  \\
\verb!\cline{!$x$\verb!-!$y$\verb!}!  &
                        Horizontale Linie von Spalte $x$ bis $y$. \\
%\verb!\multicolumn{!\textit{n}\verb!}{!\textit{cols}\verb!}{!\textit{text}\verb!}! \\
%        & Teilt eine Zelle in \textit{n} Spalten, mit den Spalteneigenschaften \textit{cols}.
\end{tabular}


\section{Mathematikmodus}
Formeln im Fließtext mit \texttt{\$formel\$}. Abgesetzte Formeln mit
\verb!\begin{equation}! oder anderer Matheumgebung.



\begin{tabular}{@{}l@{\hspace{1em}}l@{\hspace{2em}}l@{\hspace{1em}}l@{}}
Hoch$^{x}$       &
\verb!^{x}!             &  
Tief$_{x}$         &
\verb!_{x}!             \\  
$\sqrt[n]{x}$           &
\verb!\sqrt[n]{x}!      &
$\frac{x}{y}$           &
\verb!\frac{x}{y}!    \\
$\sum_{k=1}^n$          &
\verb!\sum_{k=1}^n!     &
$\sum\limits_{k=1}^n$          &
\verb!\sum\limits_{k=1}^n!     \\
$\int_{a}^b$         &
\verb!\int_{a}^b!    &
$\int\limits_{a}^b$         &
\verb!\int\limits_{a}^b! \\
\end{tabular}
\begin{tabular}{@{}l@{\hspace{1em}}l@{\hspace{2em}}}
$\left( \frac{X}{Y} \right)$ &
\verb!\left( \frac{X}{Y} \right)!
\end{tabular}

\subsection{Mathematikmodus Symbole}

% The ordering of these symbols is slightly odd.  This is because I had to put all the
% long pieces of text in the same column (the right) for it all to fit properly.
% Otherwise, it wouldn't be possible to fit four columns of symbols here.

\begin{tabular}{@{}l@{\hspace{1ex}}l@{\hspace{1em}}l@{\hspace{1ex}}l@{\hspace{1em}}l@{\hspace{1ex}} l@{\hspace{1em}}l@{\hspace{1ex}}l@{}}
$\leq$          &  \verb!\leq!  &
$\geq$          &  \verb!\geq!  &
$\neq$          &  \verb!\neq!  &
$\approx$       &  \verb!\approx!  \\
$\times$        &  \verb!\times!  &
$\div$          &  \verb!\div!  &
$\pm$           & \verb!\pm!  &
$\cdot$         &  \verb!\cdot!  \\
$^{\circ}$      & \verb!^{\circ}! &
$\circ$         &  \verb!\circ!  &
$\prime$        & \verb!\prime!  &
$\cdots$        &  \verb!\cdots!  \\
$\infty$        & \verb!\infty!  &
$\neg$          & \verb!\neg!  &
$\wedge$        & \verb!\wedge!  &
$\vee$          & \verb!\vee!  \\
$\supset$       & \verb!\supset!  &
$\forall$       & \verb!\forall!  &
$\in$           & \verb!\in!  &
$\rightarrow$   &  \verb!\rightarrow! \\
$\subset$       & \verb!\subset!  &
$\exists$       & \verb!\exists!  &
$\notin$        & \verb!\notin!  &
$\Rightarrow$   &  \verb!\Rightarrow! \\
$\cup$          & \verb!\cup!  &
$\cap$          & \verb!\cap!  &
$\mid$          & \verb!\mid!  &
$\Leftrightarrow$   &  \verb!\Leftrightarrow! \\
$\dot a$        & \verb!\dot a!  &
$\hat a$        & \verb!\hat a!  &
$\bar a$        & \verb!\bar a!  &
$\tilde a$      & \verb!\tilde a!  \\

$\alpha$        &  \verb!\alpha!  &
$\beta$         &  \verb!\beta!  &
$\gamma$        &  \verb!\gamma!  &
$\delta$        &  \verb!\delta!  \\
$\epsilon$      &  \verb!\epsilon!  &
$\zeta$         &  \verb!\zeta!  &
$\eta$          &  \verb!\eta!  &
$\varepsilon$   &  \verb!\varepsilon!  \\
$\theta$        &  \verb!\theta!  &
$\iota$         &  \verb!\iota!  &
$\kappa$        &  \verb!\kappa!  &
$\vartheta$     &  \verb!\vartheta!  \\
$\lambda$       &  \verb!\lambda!  &
$\mu$           &  \verb!\mu!  &
$\nu$           &  \verb!\nu!  &
$\xi$           &  \verb!\xi!  \\
$\pi$           &  \verb!\pi!  &
$\rho$          &  \verb!\rho!  &
$\sigma$        &  \verb!\sigma!  &
$\tau$          &  \verb!\tau!  \\
$\upsilon$      &  \verb!\upsilon!  &
$\phi$          &  \verb!\phi!  &
$\chi$          &  \verb!\chi!  &
$\psi$          &  \verb!\psi!  \\
$\omega$        &  \verb!\omega!  &
$\Gamma$        &  \verb!\Gamma!  &
$\Delta$        &  \verb!\Delta!  &
$\Theta$        &  \verb!\Theta!  \\
$\Lambda$       &  \verb!\Lambda!  &
$\Xi$           &  \verb!\Xi!  &
$\Pi$           &  \verb!\Pi!  &
$\Sigma$        &  \verb!\Sigma!  \\
$\Upsilon$      &  \verb!\Upsilon!  &
$\Phi$          &  \verb!\Phi!  &
$\Psi$          &  \verb!\Psi!  &
$\Omega$        &  \verb!\Omega!  
\end{tabular}
\footnotesize

%\subsection{Special symbols}
%\begin{tabular}{@{}ll@{}}
%$^{\circ}$  &  \verb!^{\circ}! Ex: $22^{\circ}\mathrm{C}$: \verb!$22^{\circ}\mathrm{C}$!.
%\end{tabular}

\subsection{Mathematik Umgebungen}
\settowidth{\MyLen}{\texttt{.begin.quotation.}}
\begin{tabular}{@{}p{\the\MyLen}%
                @{}p{\linewidth-\the\MyLen}@{}}
\verb!\begin{equation}!    &  Normale nummerierte Formel. \\
\verb!\begin{align}!      &  Mehrzeilige nummerierte Formeln. \\
\verb!\begin{multline}!  &  Eine mehrzeilige Formel.
\end{tabular}

Nummerierung unterdrücken mit einem \verb!*! nach der Umgebung.

\section{Literaturverzeichnis und Zitate}
Wenn \BibTeX verwendet wird muss folgendermaßen kompiliert werden: \texttt{latex}$\rightarrow$\texttt{bibtex}$\rightarrow$\texttt{latex}$\rightarrow$\texttt{latex}

\subsection{Zitieren}
\settowidth{\MyLen}{\texttt{.shortciteN.key..}}
\begin{tabular}{@{}p{\the\MyLen}@{}p{\linewidth-\the\MyLen}@{}}
\verb!\cite{!\textit{key}\verb!}!       &
        Alle Autoren und Jahr. (Watson and Crick 1953) \\
\verb!\citeA{!\textit{key}\verb!}!      &
        Alle Autoren. (Watson and Crick) \\
\verb!\citeN{!\textit{key}\verb!}!      &
        Alle Autoren und Jahr. Watson and Crick (1953) \\
\verb!\citeyear{!\textit{key}\verb!}!   &
        Nur Jahr zitieren. (1953) \\
\end{tabular}

Für \BibTeX{} nur mit \verb!\cite{!\textit{key}\verb!}! zitieren.

\subsection{\BibTeX\ Eintragsstruktur}
\settowidth{\MyLen}{\texttt{.mastersthesis.}}
\begin{tabular}{@{}p{\the\MyLen}@{}p{\linewidth-\the\MyLen}@{}}
\verb!@article!         &  Journal- oder Magazinartikel. \\
\verb!@book!            &  Buch mit Verleger. \\
\verb!@booklet!         &  Buch ohne Verleger. \\
\verb!@conference!      &  Artikel in Konferenzprotokoll. \\
\verb!@inbook!          &  Teil oder Ausschnitte eines Buches. \\
\verb!@incollection!    &  Ausschnitt eines Buches mit eigenem Titel. \\
%\verb!@manual!          &  Technical documentation. \\
%\verb!@mastersthesis!   &  Master's thesis. \\
\verb!@misc!            &  Wenn sonst nichts passt. \\
\verb!@phdthesis!       &  Dissertation. \\
\verb!@proceedings!     &  Konferenzprotokoll. \\
\verb!@techreport!      &  Technischer Bericht. \\
\verb!@unpublished!     &  Unveröffentlicht. \\
\end{tabular}

\subsection{\BibTeX\ Felder}
\settowidth{\MyLen}{\texttt{organization.}}
\begin{tabular}{@{}p{\the\MyLen}@{}p{\linewidth-\the\MyLen}@{}}
\verb!address!         &  Adresse des Verlages. Bei großen Verlagen nicht notwendig.  \\
\verb!author!           &  Name des Autors. \\
\verb!booktitle!        &  Titel eines teilzitierten Buches. \\
\verb!chapter!          &  Kapitel- oder Abschnittsnummer. \\
\verb!edition!          &  Auflage des Buches. \\
\verb!editor!           &  Name des Redakteurs. \\
\verb!institution!      &  Sponsor des tech.\ Berichtes. \\
\verb!journal!          &  Journal Name. \\
\verb!key!              &  Manueller Zitatschlüssel. \\
\verb!month!            &  Veröffentlichungsmonat. 3-Zeichenabkürzung. \\
\verb!note!             &  Zusätzliche Informationen. \\
\verb!number!           &  Magazin- oder Journalnummer. \\
\verb!organization!     &  Konferenzsponsor. \\
\verb!pages!            &  Seiten (\verb!2,6,9--12!). \\
\verb!publisher!        &  Verlagsname. \\
\verb!school!           &  Hochschulname (Dissertation, etc.). \\
\verb!series!           &  Buchserie. \\
\verb!title!            &  Titel. \\
\verb!type!             &  Art des tech. Berichtes, bsp. ``Research Note''. \\
\verb!volume!           &  Band des Journals oder Buches. \\
\verb!year!             &  Publikationsjahr. \\
\end{tabular}
Nicht alle Felder werden benötigt (siehe Beispiel).

\subsection{Gängige \BibTeX\ Stile}
\begin{tabular}{@{}l@{\hspace{1em}}l@{\hspace{3em}}l@{\hspace{1em}}l@{}}
\verb!abbrv!    &  Standard &
\verb!abstract! &  \texttt{alpha} mit Auszug \\
\verb!alpha!    &  Standard &
\verb!apa!      &  APA \\
\verb!plain!    &  Standard &
\verb!unsrt!    &  Unsortiert \\
\end{tabular}

Das \LaTeX\ Dokument sollte die folgenden zwei Befehl vor 
\verb!\end{document}! stehen haben. \verb!bibfile.bib! ist der Name der
\BibTeX\-Datei.
\begin{verbatim}
\bibliographystyle{plain}
\bibliography{bibfile}
\end{verbatim}

\section{\BibTeX\ Beispiel}
Die \BibTeX\ Datenbank ist in der Datei
\textit{file}\texttt{.bib} abgespeichtert, sie wird mit dem Befehl \verb!bibtex file! verarbeitet. 
\begin{verbatim}
@String{N = {Na\-ture}}
@Article{WC:1953,
  author  = {James Watson and Francis Crick},
  title   = {A structure for Deoxyribose Nucleic Acid},
  journal = N,
  volume  = {171},
  pages   = {737},
  year    = 1953
}
\end{verbatim}


\section{Beispiel für deutsches \LaTeX\ Dokument}
\begin{verbatim}
\documentclass[12pt,a4paper,parskip]{scrreprt}
\usepackage[utf8]{inputenc}
\usepackage[T1]{fontenc}
\usepackage[ngerman]{babel}

\title{Vorlage}
\author{Name}

\begin{document}
\maketitle

\section{Überschrift}
\subsection*{Überschrift ohne Nummer}
text \textbf{bold text} text. Hier eine Formel: $2+2=5$
\subsection{Überschrift}
text \emph{hervorgehoben} text. \cite{WC:1953}
entdeckte die Struktur von DNA

Eine Tabelle:
\begin{table}
\begin{tabular}{|l|c|r|}
\hline
Daten  &  erste  &  Reihe \\
Daten &  zweite  &  Reihe \\
\hline
\end{tabular}
\caption{Tabellenunterschrift}
\label{ex:table}
\end{table}

Eine nummerierte Tabelle \ref{ex:table}.
\end{document}
\end{verbatim}



\rule{0.3\linewidth}{0.25pt}
\scriptsize

Copyright \copyright\ 2006 Winston Chang\\

% Should change this to be date of file, not current date.
\verb!$Revision: 1.13 $, $Date: 2008/05/29 06:11:56 $.!

http://www.stdout.org/$\sim$winston/latex/

Deutsche Überarbeitung Tammo Schwindt, DHBW Mosbach

\end{multicols}
\end{document}
